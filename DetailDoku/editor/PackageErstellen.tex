\section{Package erzeugen}
\label{sec:editor.allg.packages}

Ein Package kann über das Menü Datei - Package erzeugen... erstellt werden.

Wird beim Suchen der abhängigen Dateien erkannt, dass eine Datei fehlt wird am Ende des Vorgangs eine Liste mit den fehlenden Dateien gezeigt.\abversion{2.9}

\subsection{Diese Dateien löschen / Packages deinstallieren}

Über die Schaltfläche Datei / Package hinzufügen lassen sich Dateien
hinzufügen, die bei der Installation des Package gelöscht werden sollen.
Hierbei ist es wichtig, dass nur Dateien hinzugefügt werden dürfen, die
sicher nicht von anderen Packages benutzt werden.

Seit Version 2.8.2 können auch ganze Packages hinzugefügt werden. Wird
ein Package hinzugefügt, wird dieses Package beim Benutzer
deinstalliert. Die Identifikation des Package erfolgt dabei über eine
Prüfsumme und nicht über den Dateinamen des Package. Ändert man etwas am
Inhalt des Package, ändert sich auch dessen Prüfsumme.

Es wird empfohlen bei neuen Versionen einer Strecke oder Führerstand die
alten Versionen des Package in diese Package deinstallieren Liste
hinzuzufügen. So werden beim Benutzer stets die nicht mehr benötigten
Dateien gelöscht
