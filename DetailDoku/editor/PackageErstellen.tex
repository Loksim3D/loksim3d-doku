\section{Package erzeugen}
\label{sec:editor.allg.packages}

Ein Package kann über das Menü Datei - Package erzeugen... erstellt werden.

Wird beim Suchen der abhängigen Dateien erkannt, dass eine Datei fehlt wird am Ende des Vorgangs eine Liste mit den fehlenden Dateien gezeigt.\abversion{2.9}

\subsection{Diese Dateien löschen / Packages deinstallieren}

Über die Schaltfläche Datei / Package hinzufügen lassen sich Dateien
hinzufügen, die bei der Installation des Package gelöscht werden sollen.
Hierbei ist es wichtig, dass nur Dateien hinzugefügt werden dürfen, die
sicher nicht von anderen Packages benutzt werden.

Seit Version 2.8.2 können auch ganze Packages hinzugefügt werden. Wird
ein Package hinzugefügt, wird dieses Package beim Benutzer
deinstalliert. Die Identifikation des Package erfolgt dabei über eine
Prüfsumme und nicht über den Dateinamen des Package. Ändert man etwas am
Inhalt des Package, ändert sich auch dessen Prüfsumme.

Es wird empfohlen bei neuen Versionen einer Strecke oder Führerstand die
alten Versionen des Package in diese Package deinstallieren Liste
hinzuzufügen. So werden beim Benutzer stets die nicht mehr benötigten
Dateien gelöscht

\subsection{l3dmeta Dateien}\abversion{2.9.6}
\label{sec:editor.allg.packages.l3dmeta}
Zur Definition von Eigenschaften von Dateien die nicht im Loksim XML-Format abgespeichert werden (wie zum Beispiel Texturen oder 3D-Objekte in externen Formaten) können
.l3dmeta Dateien angelegt werden, die beim Erzeugen von Packages ausgewertet werden.

Beispiel: In einem Package für eine Loksim-Strecke wird das Objekt 
\textbackslash Objekte\textbackslash\linebreak[0]Bahnhof\textbackslash MaxM\textbackslash BhfX.dae
 verwendet. 
Der Autor \emph{Max Mustermann} möchte 
explizit ausdrücken, dass dieses Objekt von ihm erstellt wurde und unter der Lizenz \emph{Loksim-AddonLizenz-2019-08} steht.

Um dies zu erreichen, ist es möglich eine Datei 
\textbackslash Objekte\textbackslash\linebreak[0]Bahnhof\textbackslash\linebreak[0]MaxM\textbackslash BhfX.dae.l3dmeta 
mit folgendem Inhalt zu erstellen:

\begin{lstlisting}[language=XML]
<?xml version="1.0"?>
<METADATA>
    <Props FileAuthor="Max Mustermann" 
    FileLicense="\Lizenzen\Loksim-AddonLizenz-2019-08.pdf" />
</METADATA>
\end{lstlisting}

Wird nun ein Package erzeugt welches die Datei 
\textbackslash Objekte\textbackslash\linebreak[0]Bahnhof\textbackslash\linebreak[0]MaxM\textbackslash BhfX.dae, 
wird automatisch auch die Datei 
\textbackslash Objekte\textbackslash\linebreak[0]Bahnhof\textbackslash\linebreak[0]MaxM\textbackslash\linebreak[0]BhfX.dae.l3dmeta
und 
\textbackslash Lizenzen\textbackslash\linebreak[0]Loksim-AddonLizenz-2019-08.pdf 
ins Package mit aufgenommen. Somit ist stets nachvollziehbar, unter welcher Lizenz das Objekt steht
und wer der Autor dieser Datei ist.