\section{Logische Ausdrücke}
\label{sec.editor.obj.logischeausdruecke}

Bei Streckenobjekten, Gruppenobjekten und Streckensounds können logische
Ausdrücke definiert werden, welche bestimmen ob das Objekt angezeigt
wird oder nicht (bzw der Sound abgespielt wird oder nicht).

In logischen Ausdrücken können Variablen (Operanden) mittels Operatoren
verknüpft werden: Wird dieser Ausdruck zu einem Wert gleich 0
ausgewertet, so wird der gesamte Ausdruck als falsch gewertet. Ist der
berechnete Wert ungleich 0, so ist der gesamte Ausdruck wahr.

\infobox{Dieses Verhalten, die möglichen Operatoren und Priorität
entsprechen der C/++ Programmiersprache.}

\subsection{Operanden (Variablen)}

\subsubsection{Konstante}
Es können folgende Arten von Konstanten als Operatoren verwendet werden:
\begin{description}
\item[Ganzzahlen] Ganze Zahlen im Bereich $-(2^{31}) + 2$ bis $2^{31} - 1$
\item[Zeitpunkte] Zeitangaben im Format hh:mm::ss oder hh:mm (Stunden:Minuten:Sekunden). Diese Art von  Angabe wird bei der Auswertung in Sekunden nach 00:00 Uhr umgerechnet.
\item[Zeichenketten] Zeichenketten - auch Strings oder alphanumerische Werte genannt - sind mit einem Anführungszeichen am Beginn und Ende gekennzeichnet ("<Wert>"). Auch Zeichenketten werden mit einer Hash-Funktion in ganze Zahlen umgewandelt. Es kann deshalb in sehr unwahrscheinlichen Fällen vorkommen, dass zwei unterschiedliche Zeichenketten in die gleiche Ganzzahl umgewandelt werden. Bei Zeichenketten macht nur der Vergleich auf (Un)gleichheit Sinn; Vergleiche wie welche Zeichenkette ist die ''größere'' von zwei werden von Loksim nicht unterstützt und das Ergebnis kann sich von Version zu Version ändern.\abversion{2.9.2}
\end{description}

\subsubsection{Variablen}
\label{sec:editor-obj-logischeausdruecke-vars}

Zusätzlich kann auf diverse Daten per Namen zugegriffen werden. Diese
Daten werden immer mittels \emph{Namensraum::Variablenname}
angesprochen.

Derzeit sind folgende Namensräume bzw Variablen definiert:

\begin{description}
\item[\texttt{Sim}]
Dieser Namensraum enthält alle Eigenschaften die in früheren Versionen über die \emph{nur (un)sichtbar bei} Optionen verfügbar waren.\abversion{2.9}
Die folgende Auflistung enthält derzeit nur einen Bruchteil der verfügbaren Variablen, alle möglichen Variablen sind über das entsprechende Auswahlfeld im Objekteditor ersichtlich
\begin{description}
\item[\texttt{VSigKennzahlKleiner}]
Diese Variable ist TRUE, wenn das Haupt- und das Vorsignal am gleichen Standort Kennziffern haben und die Vorsignalkennziffer kleiner als die (Haupt-)Signalkennziffer ist. In allen andern Fällen ist VorSigKennzahlKleiner FALSE.
\end{description}


\item[\texttt{WetterVars}]
Ermöglicht den Zugriff auf vom Benutzer definierte Variablen in
Wetterdateien
\item[\texttt{FahrplanVars}]
Ermöglicht den Zugriff auf vom Benutzer definierte Variablen in
Fahrplänen
\item[\texttt{FahrplanDaten}]
Ermöglicht den Zugriff auf diverse Daten des Fahrplans

\begin{description}
\item[\texttt{Abfahrtszeit("\textless{}halt\textgreater{}")}]
geplante Abfahrtszeit des Halts \textless{}halt\textgreater{} in
Sekunden
\item[\texttt{Ankunftszeit("\textless{}halt\textgreater{}")}]
geplante Ankunftszeit des Halts \textless{}halt\textgreater{} in
Sekunden
\item[\texttt{Halt("\textless{}halt\textgreater{}")}]
Wahr falls der Zug an dieser Haltestelle hält\abversion{2.9}
\item[\texttt{BedarfshaltBahnsteig("\textless{}halt\textgreater{}")}]
Wahr falls der Halt ein Bedarfshalt ist und der Haltanzeiger am Bahnsteig aufleuchtet\abversion{2.9}
\item[\texttt{BedarfshaltZug("\textless{}halt\textgreater{}")}]
Wahr falls der Halt ein Bedarfshalt ist und der Haltanzeiger im Führerstand aufleuchtet\abversion{2.9}
\item[\texttt{LastHalt}]
Letzter Halt im Fahrplan der keine Zugfolgestelle bzw. Betriebshalt ist\abversion{2.9.3}
\end{description}

Innerhalb der Anführungszeichen muss der exakte Name des Halts
aufgeführt werden. Sollte der Haltname selbst ein Anführungszeichen
enthalten, muss dieses durch \textbackslash{}" ersetzt werden. Der Halt
St. Pölten "Hbf" wird damit zu St. Pölten
\textbackslash{}"Hbf\textbackslash{}"
\item[\texttt{WetterDaten}]
Ermöglicht den Zugriff auf vom Programm berechnete Daten zum Wetter

\begin{description}
\item[\texttt{HelligkeitProzent}]
Helligkeit im Bereich 0-100
\end{description}

\item[\texttt{Simulation}]
Diverse Eigenschaften der Simulation
\begin{description}
\item[\texttt{ZeitInSekunden}]
Seit 00:00 vergangene Sekunden
\end{description}

\item[\texttt{Sonstige}]
Sammelkategorie für den Rest

\begin{description}
\item[\texttt{Zufall}]
Liefert einen zufälligen Wert im Bereich {[}0;Sehr große Ganzahl{[}
\item[\texttt{Zuglaenge}]
Liefert die Länge des Zugs in Metern (inklusive Triebfahrzeug)
\item[\texttt{ZufallGruppenObjekt}]
Liefert einen zufälligen Wert im Bereich {[}0;Sehr große Ganzzahl{[},
welcher für das gesamte Gruppenobjekt gleich ist. Mit dieser Variable
ist es möglich, mehrere Teilobjekte eines Gruppenobjekts zufällig ein-
oder auszublenden.

Der Einsatz dieser Variablen macht aus diesem Grund auch nur innerhalb
eines Gruppenobjekts Sinn
\end{description}
\end{description}

Wird auf eine Variable zugegriffen, die nicht definiert ist, so wird
\emph{0} (falsch) zurückgeliefert

\paragraph{Testen im LoksimEdit}

Zu Testzwecken können im LoksimEdit unter Ansicht - Erweiterte
3D-Ansichtsoptionen oder Strg+Q \emph{sämtliche} (auch von der Simulation
vorgegeben, wie ZeitInSekunden) Variablen mit benutzerdefinierten Werten
belegt werden. Damit kann man Testen, wie die Anzeige mit dieser
Variablenbelegung aussieht

\subsection{Operatoren}

Operatoren verknüpfen die Operanden miteinander um das Endergebnis zu
berechnen. Auflistung der verfügbaren Operatoren nach
\emph{absteigender} Priorität angeordnet (gleiche Hintergrundfarbe
bedeutet gleiche Priorität):

\begin{longtable}[c]{@{}lll@{}}
\hline\noalign{\medskip}
\begin{minipage}[b]{0.08\columnwidth}\raggedright
Opera-tor
\end{minipage} & \begin{minipage}[b]{0.77\columnwidth}\raggedright
Bedeutung
\end{minipage} & \begin{minipage}[b]{0.15\columnwidth}\raggedright
Alternative Syntax
\end{minipage}
\\\noalign{\medskip}
\hline\noalign{\medskip}
\begin{minipage}[t]{0.08\columnwidth}\raggedright
!
\end{minipage} & \begin{minipage}[t]{0.77\columnwidth}\raggedright
logisches \emph{Nicht}, Negation: Wandelt einen Wert ungleich 0 in 0 um
bzw umgekehrt
\end{minipage} & \begin{minipage}[t]{0.15\columnwidth}\raggedright
not
\end{minipage}
\\\noalign{\medskip}
\begin{minipage}[t]{0.08\columnwidth}\raggedright
*
\end{minipage} & \begin{minipage}[t]{0.77\columnwidth}\raggedright
\emph{Multiplikation}
\end{minipage} & \begin{minipage}[t]{0.15\columnwidth}\raggedright
\end{minipage}
\\\noalign{\medskip}
\begin{minipage}[t]{0.08\columnwidth}\raggedright
/
\end{minipage} & \begin{minipage}[t]{0.77\columnwidth}\raggedright
ganzzahlige \emph{Division}
\end{minipage} & \begin{minipage}[t]{0.15\columnwidth}\raggedright
\end{minipage}
\\\noalign{\medskip}
\begin{minipage}[t]{0.08\columnwidth}\raggedright
\%
\end{minipage} & \begin{minipage}[t]{0.77\columnwidth}\raggedright
\emph{Modulo} (Rests der Division)
\end{minipage} & \begin{minipage}[t]{0.15\columnwidth}\raggedright
\end{minipage}
\\\noalign{\medskip}
\begin{minipage}[t]{0.08\columnwidth}\raggedright
+
\end{minipage} & \begin{minipage}[t]{0.77\columnwidth}\raggedright
\emph{Addition}
\end{minipage} & \begin{minipage}[t]{0.15\columnwidth}\raggedright
\end{minipage}
\\\noalign{\medskip}
\begin{minipage}[t]{0.08\columnwidth}\raggedright
-
\end{minipage} & \begin{minipage}[t]{0.77\columnwidth}\raggedright
\emph{Subtraktion}
\end{minipage} & \begin{minipage}[t]{0.15\columnwidth}\raggedright
\end{minipage}
\\\noalign{\medskip}
\begin{minipage}[t]{0.08\columnwidth}\raggedright
\textless{}
\end{minipage} & \begin{minipage}[t]{0.77\columnwidth}\raggedright
logischer Vergleich: \emph{kleiner}
\end{minipage} & \begin{minipage}[t]{0.15\columnwidth}\raggedright
\end{minipage}
\\\noalign{\medskip}
\begin{minipage}[t]{0.08\columnwidth}\raggedright
\textless{}=
\end{minipage} & \begin{minipage}[t]{0.77\columnwidth}\raggedright
logischer Vergleich: \emph{kleiner oder gleich}
\end{minipage} & \begin{minipage}[t]{0.15\columnwidth}\raggedright
\end{minipage}
\\\noalign{\medskip}
\begin{minipage}[t]{0.08\columnwidth}\raggedright
\textgreater{}
\end{minipage} & \begin{minipage}[t]{0.77\columnwidth}\raggedright
logischer Vergleich: \emph{größer}
\end{minipage} & \begin{minipage}[t]{0.15\columnwidth}\raggedright
\end{minipage}
\\\noalign{\medskip}
\begin{minipage}[t]{0.08\columnwidth}\raggedright
\textgreater{}=
\end{minipage} & \begin{minipage}[t]{0.77\columnwidth}\raggedright
logischer Vergleich: \emph{größer oder gleich}
\end{minipage} & \begin{minipage}[t]{0.15\columnwidth}\raggedright
\end{minipage}
\\\noalign{\medskip}
\begin{minipage}[t]{0.08\columnwidth}\raggedright
==
\end{minipage} & \begin{minipage}[t]{0.77\columnwidth}\raggedright
logischer Vergleich auf \emph{Gleichheit}
\end{minipage} & \begin{minipage}[t]{0.15\columnwidth}\raggedright
\end{minipage}
\\\noalign{\medskip}
\begin{minipage}[t]{0.08\columnwidth}\raggedright
!=
\end{minipage} & \begin{minipage}[t]{0.77\columnwidth}\raggedright
logischer Vergleich \emph{ungleich}
\end{minipage} & \begin{minipage}[t]{0.15\columnwidth}\raggedright
\textless{}\textgreater{}
\end{minipage}
\\\noalign{\medskip}
\begin{minipage}[t]{0.08\columnwidth}\raggedright
\&\&
\end{minipage} & \begin{minipage}[t]{0.77\columnwidth}\raggedright
logische Verknüpfung \emph{und}; liefert wahr, falls beide Operanden
wahr sind
\end{minipage} & \begin{minipage}[t]{0.15\columnwidth}\raggedright
and
\end{minipage}
\\\noalign{\medskip}
\begin{minipage}[t]{0.08\columnwidth}\raggedright
\textbar{}\textbar{}
\end{minipage} & \begin{minipage}[t]{0.77\columnwidth}\raggedright
logische Verknüpfung \emph{oder}; liefert wahr, falls einer der
Operanden wahr ist
\end{minipage} & \begin{minipage}[t]{0.15\columnwidth}\raggedright
or
\end{minipage}
\\\noalign{\medskip}
\hline
\noalign{\medskip}
\caption{Operatoren in logischen Ausdrücken}
\end{longtable}

\subsection{Funktionen}

\subsubsection{\texttt{Funktionen::TimeDif()}}

Diese Funktion berechnet die Zeitdifferenz zwischen zwei Zeitpunkten und
beachtet dabei die Möglichkeit des Tageswechsels. Die genau Logik der
Funktion lautet:

\begin{lstlisting}
    int TimeDif(Zeit1, Zeit2)
    {
        Falls (Zeit1 < Zeit2)
        {
            Ergebnis = Zeit1 + SEKUNDEN_PRO_TAG - Zeit2
        }
        Sonst
		{
			Ergebnis = Zeit1 - Zeit2
		}
        return Ergebnis
    }       
\end{lstlisting}

\subsection{Beispiele}

Um das ganze etwas klarer zu machen, ein paar Anwendungsfälle:

Ein Objekt soll nur zwischen 22:00 und 05:00 Uhr angezeigt werden, oder
wenn die Helligkeit kleiner als 20 \% ist:

\begin{lstlisting}
Simulation::ZeitInSekunden >= 22:00 ||
        Simulation::ZeitInSekunden <= 05:00 || 
        WetterDaten::HelligkeitProzent < 20
\end{lstlisting}

Ein Sound soll nur zur vollen Stunde abgepielt werden, aber nicht nach
20:00 bzw vor 06:00 Uhr:

\begin{lstlisting}
Simulation::ZeitInSekunden % (60 * 60) <= 2 &&
        !(Simulation::ZeitInSekunden > 20:00 ||
        Simulation::ZeitInSekunden < 06:00)
\end{lstlisting}

Eine optimierte, gleichwertige Version wäre:

\begin{lstlisting}
Simulation::ZeitInSekunden % 3600 <= 2 &&
        Simulation::ZeitInSekunden <= 20:00 &&
        Simulation::ZeitInSekunden <= 06:00 )
\end{lstlisting}
				
Erklärung: Zu jeder vollen Stunde ist die ZeitInSekunden durch 3600 ohne Rest
teilbar. Dies bedeutet ZeitInSekunden \% 3600 ergibt 0. Wenn man nun
dies exakt mit 0 vergleicht, würde der Sound nur abgespielt, falls es
exakt die volle Stunde ist. Für uns ist jedoch ein kleiner Spielraum von
2 Sekunden ok, also wird mit \textless{}= 2 geprüft, sodass der Sound
auch 1 oder 2 Sekunden nach der vollen Stunde gestartet werden kann

Ein Objekt soll mit 60\%iger Wahrscheinlichkeit angezeigt werden:
\texttt{Sonstige::Zufall \% 100 \textless{} 60} Erklärung: Zufall
liefert eine zufällige Zahl im Bereich 0 bis sehr große Zahl. Rechnet
man diese Zahl modulo 100 (Rest der Division der Zahl durch 100) bekommt
man eine zufällige Zahl im Bereich {[}0;100{[} In 60\% der Fälle ist
diese Zahl kleiner als 60, und in den anderen 40 Prozent größer gleich
60. Also hat man genau die 60\%ige Wahrscheinlichkeit

Differenz zur Abfahrtszeit des Halts \emph{h0 ''ui''}

\begin{lstlisting}
Funktionen::TimeDif( 
        FahrplanDaten::Abfahrtszeit("h0 \"ui\""), 
        Simulation::ZeitInSekunden) / 60)
\end{lstlisting}

Ein Objekt soll nur angezeigt werden, falls der - händisch in den Fahrplanvariablen - gesetzte Endbahnhof ''Hintertupfingen'' ist:\abversion{2.9.2}

\begin{lstlisting}
FahrplanVars::Endbahnhof == "Hintertupfingen"
\end{lstlisting}

\subsection{Einfluss auf Performance}

Bei großflächigem Einsatz von logischen Ausdrücken, sollte man etwaige
Auswirkungen auf die Performance bedenken:

\begin{itemize}
\item
  Objekte bei denen \emph{Objekt in Sichtweite ändert Sichtbarkeit
  nicht} gesetzt ist, haben kaum Einfluss auf die Performance: Solche
  Ausdrücken müssen während der Simulation nur ein einziges Mal
  berechnet werden und sind anschließend fixiert
\item
  Auch konstante Ausdrücke (kein Zugriff auf Variablen außer
  Sonstige::Zufall) müssen nur ein einziges Mal berechnet werden
\item
  Sonstige logische Ausdrücke bei Objekten werden regelmäßig (mehrmals
  pro Sekunde) neu berechnet, solange sich die Objekte in sichtbarer
  Entfernung befinden
\item
  Nicht-konstante, logische Ausdrücke bei Sounds, werden "regelmäßig"
  (mehrmals pro Sekunde) neu berechnet, solange sich die Lok im
  Gültigkeitsbereich des Sounds befindet
\item
  Es werden keinerlei Optimierungen der Ausdrücke vorgenommen!
\end{itemize}