\subsubsection{Signaloptionen}
\label{subsubsec.editor.gleis.gleiseigenschaften.signaloptionen}

\paragraph{Gleisabhängiges Zusatzsignal}
Diese Funktion kann für Zusatzsignale benutzt werden, die je nach Gleis ein bestimmtes Font anzeigen sollen, ähnlich dem Zs2. Bei diesem Zusatzsignal wird aber auf der KBS-Strecke in Fahrrichtung rückwärts nach einem Eintrag für dieses Zusatzsignal gesucht. Mit dem Setzen eines Referenzzeichens (Buchstabe, Zahl) kann der entsprechende Font angesteuert werden. Es wird ab Signal maximal 1000 m rückwärts nach der Zusatzsignalreferenz gesucht.

Einrichten eines solchen Zusatzsignals:
\begin{itemize}
\item
In einem Gruppenobjekt den entsprechenden Schriftzug wählen und beim Popup-Menu  "Text aus Eigenschaft der Gruppe" den Eintrag "GlAbhZSig" auswählen.
\item
In den Eigenschaften einer Strecke an der gewünschten Position eine neue Signaloption definieren und bei "Gleisabhängiges Zusatzsignal" das Häckchen setzen. Beim Eintrag Referenzzeichen ein Zeichen (Buchstabe, Zahl) eintragen, das dem gewünschten Fontzeichen entspricht.
\end{itemize}