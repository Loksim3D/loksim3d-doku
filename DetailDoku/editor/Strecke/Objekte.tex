\section{Streckenobjekte}

\subsection{Wiederholung an Achse}\abversion{2.9}
\label{sec:editor-strecke-obj-achsewiederholung}
Über den Button \emph{Achse Wiederholung...} besteht die Möglichkeit, Objekte an einer benutzerdefinierten Achse zu wiederholen. Dies macht nur Sinn, wenn bei \emph{Anzahl} mehr als ein Objekt eingetragen ist. Wird die Funktion aktiviert, werden die Objekte nicht entlang des Streckenverlaufs wiederholt, sondern entlang der benutzerdefinierten Achse. Die Achse wird dabei durch einen x/y/z Winkel definiert, der relativ zum aktuellen Gleis verwendet wird.

Die Funktion kann beispielsweise verwendet werden, um einfach eine Baumreihe die in 90\degree von der Strecke weg verläuft, zu erstellen.
