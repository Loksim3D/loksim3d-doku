\section{Indusi/PZB}

In Version 2.8.2 wurde eine Überarbeitung der PZB90 eingebaut. Dieser
Abschnitt bezieht sich derzeit, nur auf die PZB-Versionen, welche von
dieser Überarbeitung betroffen sind.

\subsection{Art der Indusi}

Folgende PZB-Arten wurden mit Version 2.8.2 erneuert:

\begin{itemize}
\itemsep1pt\parskip0pt\parsep0pt
\item
  PZB90 I60R
\item
  PZB90 I60/ER24
\item
  PZB90 PZ80R
\item
  PZB90 I80
\end{itemize}

Die PZB Version kann für diese Typen im \hyperref[sec.editor.pzb.einstellungen]{PZB Einstellungsdialog} geändert
werden

Eine detaillierte Beschreibung der Systeme kann im \href{http://www.dbnetze.com/regelwerke}{betrieblich-technischen Regelwerk der DB Netz AG}
nachgelesen werden.

Bei I60R ertönt der Sound für die Freitaste nur, wenn tatsächlich eine
Befreiung möglich ist. Bei I60/ER24 ertönt dieser Sound immer wenn die
Freitaste gedrückt ist.

Kontrolllauf und PZ80R Kontrollschalter sind derzeit nicht implementiert.

Jene Optionen die mit \emph{(veraltet)} gekennzeichnet sind, sollten bei
neuen Loks nicht mehr verwendet werden. Diese Optionen stehen nur
aufgrund Rückwärtskompatiblität zu älteren Führerständen zur Verfügung

\subsection{Sounds}

\begin{description}
\item[Indusihupe (WT, FT)]
Sound welcher bei Betätigung von Wachsamkeitstaste bzw. Freitaste
abgespielt wird. Dieser wird in einer Schleife abgespielt, solange die
entsprechende Taste gedrückt ist.
\item[Indusibefehl]
Sound welcher in einer Schleife abgespielt wird, solange die
Befehlstaste aktiv ist.
\item[Zwangsbr.-Indusi]
Dieser Sound wird in einer Schleife während einer Zwangsbremsung
abgespielt.
\item[Zwangsbr.-Indusi nur einmal]
Dieser Sound wird exakt einmal am Anfang einer Zwangsbremsung
abgespielt.
\item[Ende 500Hz Überwachung]
Sound wird am Ende einer 500Hz Überwachung abgespielt, falls sie im
restriktiven Modus endet.
\item[Überschreiten V-Pruef]
Dieser Sound wird beim Überschreiten der PZB-Prüfgeschwindigkeit
wiederholt abgespielt. Je nach PZB-Art kommt dieser Sound sofort bei
Überschreiten oder erst nach einer definierten Zeitspanne.
\end{description}

\subsection{Leuchtmelder und Anzeigen}

Die Leuchtmelder und Anzeigen werden wie andere Anzeigen im Lokeditor
definiert. Bei Einsatz von PZ80R sollte von den Leuchtmeldern nur der
Leuchtmelder Indusi 95 (LVZ grün) definiert werden, aber dafür die
Anzeige IndusiVZiel. Bei den anderen Arten sollte alle Leuchtmelder
gesetzt werden, jedoch die Anzeige IndusiVZiel nicht.

\subsection{PZB Einstellungen}
\label{sec.editor.pzb.einstellungen}

Im Lokeditor ist über das Menü Bearbeiten - PZB Einstellungen ein Dialog
abrufbar, der weitere Einstellungen für die PZB ermöglicht.

Zum Großteil geht es hier um Eigenschaften der PZB die nicht exakt aus
den uns zur Verfügung stehenden Unterlagen implementiert werden konnten
oder wo es in der Praxis konträre Erfahrungen gibt. Jedoch gibt es auch
Einstellungen die bekanntermaßen von Lok zu Lok unterschiedlich sind

\begin{description}
\item[Version]
Hier kann die Version der PZB90 ausgewählt werden. Es stehen alle
Versionen die in der Realität vewendet werden und wurden (1.5, 1.6 und
2.0) zur Verfügung
\item[Befehlstaste ist ein Schalter]
Ein Druck auf die Befehlstaste aktiviert die Befehlstaste und ein
zweiter Druck deaktiviert diese.
\item[Befehlstaste ist ein Taster]
Befehlstaste ist nur aktiv, solange die Taste gedrückt wird. Dies ist
meist bei neueren Tfz der Fall.
\item[Zeit für Wachsamkeitstaste]
Definiert die Zeit in Millisekunden innerhalb welcher nach einer 1000Hz
Beeinflussung die Wachsamkeitstaste gedrückt werden muss. Normalerweise
sind dies 4 Sekunden, im Fall der MVB jedoch 2,5 Sekunden.
\item[Befreiung aus Zwangsbremsung...]
Annahme: Bei Stillstand nach einer Zwangsbremsung aufrund einer 1000Hz
Überwachung sind bereits 700m oder mehr ab Beginn der Beeinflussung
vergangen. Nun muss die Freitaste zur Befreiung aus der ZB betätigt
werden.

Falls die Option kann gleichzeitig Befreiung aus 1000Hz Überwachung
bewirken gesetzt ist, bewirkt ein Druck auf die Freitaste die Befreiung
aus der ZB, sowie die Befreiung aus der 1000Hz Überwachung.

Ist hingegen bewirkt niemals gleichzeitig Befreiung aus einer
Überwachung gewählt, bewirkt ein Drücken der Freitaste ausschließlich
die Befreiung aus der ZB.
\item[Dauerbetätigung PZB Tasten]
Wird hier eine Option gesetzt, wird die entsprechende Taste nach der
bestimmten Zeit und/oder Entfernung unwirksam. 0 bedeutet, dass die
Taste kein Zeit- bzw. Entfernungsmaximum besitzt.

Die Richtlinien der ÖBB geben eine maximale Distanz von 225m für alle
Tasten an.
\item[Status 'befreit' wird an überlagerte Überwachung weitergegeben]
Wird diese Option gesetzt, wird der Status 'befreit' einer 1000Hz
Überwachung an eine überlagerte 1000Hz Überwachung weitergegeben.
Angenommen folgende Situation: 0m/1000Hz, 800m/Befreiung, 1000m/1000Hz,
1400m/500Hz: Ist diese Option nicht gesetzt, erfolgt am 500Hz Magnet
keine Zwangsbremsung, solange man die entsprechende Geschwindigkeit
einhält. Ist die Option gesetzt, bekommt man am 500Hz Magnet immer eine
ZB aufgrund nicht erlaubter Befreiung
\item[Sonderform]
Über diese Einstellungen können Sonderformen der PZB simuliert werden.

Derzeit ist nur die Sonderform \emph{Stadtbahn} möglich. Bei dieser
Variante gibt es neben dem Wechselblinken ein Gleichblinken, bei welchem
65km/h gefahren werden dürfen.
\end{description}

