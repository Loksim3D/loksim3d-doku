\section{Versionshistorie}
\subsection{Version 2.9.3}
\subsubsection{Neue Funktionen}
\begin{itemize}
\item Neue Optionen für \hyperref[sec:editor-gleise]{Gleise}: Keine 3D-Darstellung, senkrechte Normalenvektoren und neue Transparenzoptionen
\item \hyperref[sec:sim-optionen-special]{Fahrplanabhängige Sichtweite}
\item Interne Umstelllung der Gleiserzeugung
\item Editor: Objekteditor zeigt Fehler bei Flächen an
\item Editor: Textur im Fonteditor zoombar
\item Objekteditor: \hyperref[sec:editor-obj-l3dobj-normalen]{Normalenvektoren für runde Objekte berechnen}
\end{itemize}
\subsubsection{Kleinere Änderungen}
\begin{itemize}
\item FahrplanDaten::LastHalt auch in Sichtbarkeitsausdruecken verfügbar
\item \hyperref[sec:sim-optionen-paths]{paths.ini} kann nun auch <Registry> anstatt Pfad enthalten
\item Editor: Bessere Kennzeichnung von fehlerhaften Eingaben in einigen Dialogen
\item Editor: LastWork Ordner behält bis zu 500 Dateien auf
\item Streckeneditor: Löschen aller Objekte eines Gleis auf einmal möglich
\item Strecken- und KBS-Editor: Anzeige Positionsinformationen über Info-Button
\end{itemize}
\subsubsection{Fehlerkorrekturen}
\begin{itemize}
\item Absturz beim Laden von Objekten die Punkte oder Linien enthalten aus externen Formaten behoben
\item Abstürze bei .l3dobj Objekten mit nicht-planaren Flächen behoben
\item Verschwinden von Objekten in bestimmten Situationen behoben
\item Zulassen von sehr kleiner Streckenrotation
\item Simulator: Bei ausgeblendeten Kennziffern wird ein Limit am Hauptsignal dennoch aktiv
\item Simulator: Probleme bei LZB-Ende behoben
\item Strecken- und KBS-Editor: Absturz bei bestimmten GPAs behoben
\item Streckeneditor: Bugfix Eigenschaftsname mit Sonderzeichen (Eigenschaften Gruppenobjekte)
\end{itemize}

\subsection{Version 2.9.2}\hfill 11. August 2015
\subsubsection{Neue Funktionen}
\begin{itemize}
\item Zufallsdrehung bei Objekten auf Strecke möglich
\item \hyperref[sec.editor.obj.logischeausdruecke]{Zeichenketten in logischen Ausdrücken}
\item \hyperref[sec:editor-texturnutzung-optimieren]{Editor: Neue Funktion ''Texturnutzung optimieren''}
\item Objekteditor: Neuer Dialog für Punkte zu Fläche hinzufügen
\end{itemize}

\subsubsection{Kleinere Änderungen}
\begin{itemize}
\item Indusi-Einstellung PZ80R wird automatisch auf PZB90 PZ80R umgestellt
\item Kommandozeilen-Argument /renderstats:1 zeigt Anzahl von DrawCalls und Triangles
\item Optimierung bei externen Objektmodellen wenn kein Alphakanal in der Textur verwendet wird
\item \hyperref[sec:editor-logging]{Standardmäßiges Erstellen von Logs}
\item Reihenfolge der Gleise bei Verwendung von verknüpften BÜs nicht mehr relevant
\item Editor: FilePicture wird bei Speichern-Unter kopiert
\item Editor: Dateien die über ''Doppelklick'' im LoksimEdit geöffnet werden, erscheinen in RecentFileList
\item Editor: Beleuchtung mehr an Simulation angenähert
\item Editor: Löschen in Baumansichten mittels ''Entf''
\item Gruppenobjekteditor: Auswahl nach dem Löschen eines Objekts verbessert
\item Fonteditor: Speichern-Unter bei Fonts überschreibt existierende Texturen nicht mehr
\item PackageManager: Anzeige von Fehlern bei ''Installation Rückgängig machen''
\end{itemize}


\subsubsection{Fehlerkorrekturen}
\begin{itemize}
\item Bugfixes bei Sichtbarkeitssteuerung
\item Bugfix Rotation Streckenobjekte

\item ObjektEditor: Verschwinden von Objekten in seltenen Fällen
\item ObjektEditor: Texturkoordinaten ''Rückseite einfügen''

\item Streckeneditor: Setzen von Inhalt bei Textfeldern ohne Name bzw. bei Textfeldern mit dynamischer Sichtbarkeit
\item Streckeneditor: BÜ bei welchem der Name Teil des Namens eines andere BÜs ist, kann nun auch im GUI angelegt werden
\end{itemize}

\subsection{Version 2.9.1}\hfill 11. Jänner 2015

\subsubsection{Neue Funktionen}
\begin{itemize}
\item Streckeneditor: (Landschafts)objekte können nach Position sortiert werden
\end{itemize}

\subsubsection{Kleinere Änderungen}
\begin{itemize}
\item Optimierung Option ''Alle Texturen beim Start laden''
\item Vergrößerung von Eingabefeldern im Editor: Neu-Dialog, Rail - Höhenwerte der Bettung, Strecke - Unterbrechung der Höhenlinie
\item Performanceverbesserungen Packageinstallation
\item Behandlung nicht planarer Flächen bei Objekten die bis inkl. Version 2.9 erstellt wurden so wie in Version 2.8.3
\item Anzeige MessageBox bei fehlendem Sound im Simulator deaktiviert

\item Streckeneditor: Laden von Objekten für 2D-Ansicht erfolgt im Hintergrund
\item Streckeneditor: Graphikobjekte bei Streckeneigenschaften (Indusi-Magnete, Tafeln, etc) können nun jedes unterstützte Graphikformat sein (nicht nur .l3dgrp)
\item Streckeneditor: Button ''Rollmaterial anzeigen'' in ''Rollmaterial ausblenden'' umbenannt

\item Kursbucheditor: Standardauswahl bei neuer Verbindung im KBS-Editor optimiert
\end{itemize}

\subsubsection{Fehlerkorrekturen}
\begin{itemize}
\item ''Höhe über Landschaft'' bei global gesetzter Verschiebung in Y-Richtung
\item Gleichzeitiger Einsatz von ''Objekt mitdrehen'' und ''Zoomfaktor''
\item Texteigenschaft bei bestimmten Fonts
\item Streckenobjekte an Achse wiederholen und Verschiebung von Objekten
\item Einlesen ''2. Länge'' bei diversen Instrumenten von alten Führerständen

\item Editor: Mehrere 3D-Fenster gleichzeitig bedienen
\item Editor: Gehe-Zu-Postion Dialog aktualisiert 3D-Ansicht nicht
\item Editor: Bugfix Neu-Dialog bei geöffneter 3D-Ansicht
\item Editor: Korrektur Anzeige fehlender Dateien beim Erstellen von Packages
\item Editor: Bugfix Erstellen eines Package (Versionsinfo, zu löschende Dateien und Packages)

\item Objekteditor: Größenänderung Textur wird bei Drücken von ''Übernehmen'' Button erkannt
\item Objekteditor: Flächen kopieren

\item Gruppenobjekteditor: Nicht gesetzte Eigenschaft wird standardmäßig zu Wahr ''ausgewertet''
\item Gruppenobjekteditor: Objekte auf/ab
\item Gruppenobjekteditor: Punkte verschieben/zoomen Fonts

\item Streckeneditor: ''HelligkeitProzent'' nicht von manuell eingestellter Helligkeit beeinflusst
\item Streckeneditor: Erweiterte Objekteigenschaften bei Signalen
\item Streckeneditor: Einstellen von Eigenschaften verschachtelter Gruppenobjekte

\item Führerstandseditor: Undo/Redo
\end{itemize}


\subsection{Version 2.9}\hfill 9. November 2014

\subsubsection{Neue Funktionen}
\begin{itemize}
\item \hyperref[sec:editor-obj-grp]{Gruppenobjekte können selbst Gruppenobjekte enthalten}
\item \hyperref[sec:editor-obj-externe]{Unterstützung externer 3D-Objektmodellformate}
\item Möglichkeit \hyperref[sec:editor-strecke-obj-achsewiederholung]{Objekte auf Strecke an beliebiger Achse zu wiederholen}
% \item \hyperref[sec:editor-obj-textur-kachel]{Gekachelte Texturen bei Objekten}
\item \hyperref[sec:editor-obj-l3dobj-normalen]{Möglichkeit bei Objekten Normalenvektoren pro Fläche und nicht pro Punkt zu definieren}
\item Funktion \hyperref[sec:editor-obj-l3dobj-normalen]{''Berechnung Normalenvektoren''}
\item Funktion ''\hyperref[sec:editor-obj-grp-verschieben]{Gruppenobjekt}/\hyperref[sec:editor-obj-l3dobj-punkteverschieben]{Objekt} am Nullpunkt zentrieren''
\item \hyperref[sec:editor-obj-l3dobj-flaeche]{Funktion ''Rückseite einer Fläche automatisch erstellen''}
\item \hyperref[sec:editor-obj-sichtbarkeitssteuerung]{Überarbeitung und Erweiterung der Sichtbarkeitssteuerung}
\item \hyperref[sec:editor-allg-dateienueberschreiben]{Warnung beim Überschreiben fremder Dateien}
\item Neue Funktion \emph{Dateiabhängigkeiten anzeigen} im Expertenmodus des PackageManager verfügbar
\item \hyperref[sec:editor-obj-externe-konvertieren]{Konvertieren externer 3D-Objektmodellformate ins Loksim-Format}
\end{itemize}

\subsubsection{Kleinere Änderungen}
\begin{itemize}
\item Neue Systemvoraussetzung Windows Vista SP2 oder neuer 
\item \hyperref[sec:editor-obj-sichtbarkeitssteuerung-streig]{Möglichkeit Streckenmeter als Objekteigenschaft zu benutzen}
\item \hyperref[sec:editor-obj-logischeausdruecke-vars]{Variablen FahrplanDaten::Halt, BedarfshaltBahnsteig und BedarfshaltZug hinzugefügt}
\item \hyperref[sec:editor-obj-dynstr]{Neuer Operand FahrplanDaten::LastHalt für dynamische Schriften}
\item \hyperref[sec:editor-obj-dynstr-params]{Parameter von dynamischen Schriften können im Streckeneditor gesetzt werden}
\item Objektflächen müssen nicht konvex sein
\item Performanceverbesserungen
\item Objekte tauchen nicht mehr so plötzlich aus dem Nichts auf
\item \hyperref[sec:editor.allg.packages]{Fehlende Dateien beim Erstellen eines Package werden angezeigt}
\item Dialog Beleuchtung: ''Slider'' auf ''SpinBox'' geändert
\item Wegfall der Option \emph{Graphik unter x fps vereinfachen}
\item Neue Expertenoption zur Steuerung der \hyperref[sec.sim.optionen.darstellung]{Objektsichtweiten}
\item Neue Option \hyperref[sec.sim.optionen.darstellung]{Windows 8 Vollbildmodus}
\end{itemize}

\subsubsection{Fehlerkorrekturen}
\begin{itemize}
\item AFB ohne LZB-Führung nur bis 160 stellbar
\item Korrektur falsche Befreiungsmöglichkeit PZB90 1000Hz Beeinflussung
\item Zugsicherung Fahrsperre: Überwachungsgeschwindigkeit 2000Hz von 40km/h auf 10km/h verändert
\item Probleme Nachtführerstand und Standardgleis behoben
\item Korrektur OLE-Variable Wechselblinken
\item Fehler im PackageManager bei Deinstallation von Packages behoben
\item PZB-Befehl über Joystick wieder möglich
\item Editieren von Instrument Weglängenmessung korrigiert
\item Fehler beim Abspeichern benutzerdefinierter Charakteristik der Fahrstufen korrigiert
\end{itemize}

\subsection{Version 2.8.3}\hfill 2. April 2014

\subsubsection{Neue Funktionen}
\begin{itemize}
\item \hyperref[sec:editor-obj-transparenz]{Transparenz aus Alphakanal}
\item \hyperref[sec.sim.steuerung.diverses.wegmessung]{Zuglängenzähler}
\item \hyperref[paragraph.editor.gleis.gleiseigenschaften.signal.optiongruppensignal] {Signal - Option Gruppensignal}
\item \hyperref[subsubsec.editor.gleis.gleiseigenschaften.signaloptionen]{Signaloptionen - gleisabhängiges Zusatzsignal}
\item \hyperref[sec.editor.obj.logischeausdruecke]{Sichtbarkeitssteuerung - Variable VsigKennzahlKleiner}
\item \hyperref[sect.editor.lok.fst]{Nachtführerstandsbild}
\end{itemize}

\subsubsection{Kleinere Änderungen}
\begin{itemize}
\item Störschalter für PZB90
\item Verbesserung Verhalten PZB90 während LZB-Betrieb
\item Wegfall des 16px Rands für transparente \hyperref[sec:editor-obj-textur]{Texturen}
\item Dialog für den Aufruf der Hilfedateien
\item Erweiterung des Fahrtenschreibers um PZB-Tasten und LZB-Status
\item Anpassung der Uhrzeit-Werte im TCP-Protokoll an die Zusi2-Ausgaben (IDs 10, 11, 12, 50)
\item Digitale Instrumente können rechtsbündig dargestellt werden
\item Bei fehlender Fahrplandarstellung wird der erfolgreiche Halt im Buchfahrplan/EBuLa angezeigt
\item Unverriegelte Türen öffnen bei jedem Halt (auch auf freier Strecke) automatisch 
\end{itemize}

\subsubsection{Fehlerkorrekturen}
\begin{itemize}
\item Sound Ende 500Hz Überwachung
\item Fehlerkorrektur bei Überlagerung zweier 1000Hz Beeinflussungen (PZB90)
\item Korrekturen bei G- und S-Melder im LZB-Betrieb
\item Problem bei gleisabhängiger Sichtbarkeitssteuerung behoben
\end{itemize}

\subsection{Version 2.8.2a}\hfill 26. Juli 2013

\begin{itemize}
\itemsep1pt\parskip0pt\parsep0pt
\item
  Geschwindigkeiten über 160km/h mit LZB wieder möglich
\item
  Bedingtes Abspielen von Streckensounds
\end{itemize}

\subsection{Version 2.8.2}\hfill 18. Juli 2013

\begin{itemize}
\itemsep1pt\parskip0pt\parsep0pt
\item
  Bei der Darstellung des Buchfahrplans können die Spalten vier und fünf
  auf Rechtsbündig gesetzt werden.
\item
  Die Darstellung des Buchfahrplans kann auf EBuLa umgestellt werden.
  Die EBuLa-Anzeige ist nur vorbildähnlich.
\item
  2D-Fonts können sämtliche Zeichen aus Unicode enthalten
\item
  Gefahrene Km bzw. Anzahl Aufrufe werden auch pro Fahrplan gespeichert
\item
  Die Anzahl der Zeilen des Buchfahrplans im Führerstand kann nun im
  Editor gesetzt werden.
 \item 
  Komplette Neuimplementierung der PZB90
\item
  Stadtbahn PZB90  
\item
  Englische Version des Simulators
\item
  PackageManager: Deinstallation von Packages während Installation neuer
  Packages möglich
\item
  Joystick Achsen umkehrbar
\item
  Schneefall (Alpha Status) über Wetterdateien steuerbar
\end{itemize}

\begin{itemize}
\itemsep1pt\parskip0pt\parsep0pt
\item
  Der Indusimagnet des Schutzsignals wird eigenständig ausgewertet
\item
  Lüfter läuft nur bei Verwendung dynamischer Bremse nach
\item
  Lüfter schaltet mit HS aus
\item
  Anzeige Lüfterstatus verzögert anhand von Sound
\item
  Testen im LoksimEdit für Wetterdateien, Dynamische Schriften u.
  Dynamische Sichtbarkeitssteuerung verbessert
\item
  Kameraposition im (Gruppen)objekt-Editor bleibt nach Update-(Button)
  erhalten
\item
  Leerzeilen in der Buchfahplananzeige unterdrückt
\item
  Gruppenobjekt-Editor: Eigenschaften von Gruppenobjekt "überleben" ein
  Refresh der 3D-View
\item
  PackageManager: Anzeige von überschriebenen oder gelöschten
  schreibgeschützten Dateien ganz oben
\item
  Zufall pro Gruppenobjekt mit Sonstige::ZufallGruppenObjekt
\item
  CrashReport Sprache wird dynamisch anhand ausgewählter Sprache bei
  Installation bzw. PackageManager gesetzt
\item
  Sondersounds werden mit der Lautstärke für Ansagen abgespielt
\item
  Standard Sky-Datei wird nicht mehr automatisch im Fpl gesetzt
\item
  Darstellung Bedarfshalt in der Fpl-Anzeige geändert.
\item
  LoksimEdit: Datei-Öffnen Dialog enthält "Doku" Button
\item
  Verbesserte Auswahl für Standardauflösung und 3D-Treiber
\item
  Fette Überschriften im Lokeditor
\item
  Multimonitor-Support bei gleichen Treibernamen
\item
  TCP: Soll-Fahrstufe und Oberstrom
\end{itemize}

\begin{itemize}
\itemsep1pt\parskip0pt\parsep0pt
\item
  Bugfix Öffnen "Zuletzt verwendeter Datei" die nicht mehr existiert
\item
  PackageManager - Installation rückgängig machen bei schreibgeschützter
  Datei
\item
  Fehler bei Speichern unter - Textur kopieren behoben
\item
  Tippfehler und Texte im Lokeditor überarbeitet
\item
  'Verschleppte' Bezeichungen aus dem Nebengleis gefixed
\item
  Leerzeilen im Buchfahrplan Führerstand unterdrückt
\item
  Fehler in L3dEditLauncher behoben (mehrere Loksim Installationen)
\item
  Fehler in der Fahrplananzeige behoben
\item
  Mausradsteuerung
\item
  Verzerrter Sound im Stand
\item
  Korrektur Anzeige Bedarfshalt im Fst (manchmal zu früh)
\item
  Joystick mit Funktion "Kombibremshebel (inkl. Beschleunigung)" bei
  Fst. mit Kombibremshebeln verwendbar
\item
  Diverse Korrekturen bei LZB und AFB
\item
  Doppelter Nullstellzwang
\item
  Zs1 im LZB-Betrieb
\item
  AFB + Kombihebel
\end{itemize}

\paragraph{Version 2.8.1a}

07. Dezember 2012

\begin{itemize}
\itemsep1pt\parskip0pt\parsep0pt
\item
  Variable Sonstige::Zuglaenge
\item
  Senden der Daten in Fehlerberichten optimiert
\item
  Option Texturhandling standardmäßig auf "Bei Bedarf laden und nicht im
  Speicher halten"
\end{itemize}

\begin{itemize}
\itemsep1pt\parskip0pt\parsep0pt
\item
  AFB nur bis max. 160km/h ohne LZB
\item
  Bugfix Haltansagen
\item
  Bitmap bei "Font Erstellen" wird wieder automatisch generiert
\item
  PackageManager funktioniert auch auf FAT32 Partitionen
\item
  Kleinere Bugfixes im PackageManager
\end{itemize}

\paragraph{Version 2.8.1}

26. Oktober 2012

\begin{itemize}
\itemsep1pt\parskip0pt\parsep0pt
\item
  Die erste und zweite Spalte in der Fahrplandarstellungen der
  Führerstände sind jetzt als rechtsbündige Ausgabe möglich
\item
  Fonts sind jetzt wie Objekte über statische Zustandsvariablen
  schaltbar
\item
  Indusi zusätzlich: Nunmehr sind Geschwindigkeitsprüfabschnitte auch
  signalabhängig möglich
\item
  In Fahrplänen können benutzerdefinierte Variablen zur Verwendung in
  logischen Ausdrücken gesetzt werden
\item
  Trennung von Daten- und Programmverzeichnis möglich
\item
  Automatisches Erstellen von Fehlerberichten
\item
  Ein durchfahrener Halt kann als Zugfolgestelle definiert werden.
\item
  Ein Halt kann als Betriebshalt definiert werden.
\item
  Kachelung von Texturen in Skyboxen möglich
\item
  Texturen im PNG- und TGA-Format werden unterstützt
\item
  Dynamische Schriften möglich
\item
  LZB Verbesserungen: BKW und realistischere Bremskurven
\end{itemize}

\begin{itemize}
\itemsep1pt\parskip0pt\parsep0pt
\item
  Leuchtmelder Halbstufe reaktiviert
\item
  Steuerung S-Melder angepasst
\item
  Berechnung der Zuglänge im Bremszettel geändert
\item
  (Dis)connect Buttons für TCP-Anbindung eingebaut
\item
  PackageManager an kleinere Auflösungen angepasst
\item
  PackageManager kann installierte Dateien mit 'Doppelklick' sofort
  öffnen
\item
  Quadratische Texturen bei seitlichen Flächen in Skyboxen möglich und
  empfohlen
\item
  Bei Wetterauswahldialog ist standardmäßig "Zufällig" (Skybox)
  ausgewählt und nicht mehr die klassische Steuerung
\item
  Loksim(Edit) About-Dialog zeigt Verwendung von SSE2 an
\item
  Gebaeude1\_FFS von RainerH in Standard-Package 2.8.1 inkludiert
\item
  Anpassung an Benutzerkontensteuerung
\item
  Beim Package Erstellen kann man nun standardmäßig auch .txt, .pdf,
  .xps Dateien auswählen (Doku)
\item
  Über den Eigenschaften-Dialog kann man im LokimEdit jeder Datei eine
  Doku zuweisen. Für Fahrpläne und Loks wird im Loksim3D ein "Doku"
  Button zum Öffnen dieser Doku angezeigt
\item
  Joystick Slider (Schubregler) ist nutzbar
\item
  Bedarfshaltanzeiger kann im Editor mittels "Signal grün/rot" Button
  getestet werden
\item
  Bedarfshalt "immer" zu Testzwecken einstellbar
\item
  Zeitpunkt an dem Bedarfshalt-Anzeige im Fst aufleuchtet wird zufällig
  bestimmt
\item
  Installer regisistriert automatisch LoksimControl.exe
\item
  Speicherlimit für Loksim3D bzw. LoksimEdit auf 3 bzw. 4 GB angehoben
  (32 bzw. 64 Bit OS)
\item
  Erweiterung der logischen Ausdrücke

  \begin{itemize}
  \itemsep1pt\parskip0pt\parsep0pt
  \item
    FahrplanDaten::Ankunftszeit"\textless{}halt\textgreater{}"
  \item
    FahrplanDaten::Abfahrtszeit"\textless{}halt\textgreater{}"
  \item
    FahrplanVars::
  \item
    Funktionen::TimeDif(\textless{}arg1\textgreater{},
    \textless{}arg2\textgreater{})
  \end{itemize}
\item
  Neue Mausgesten bei der 3D Vorschau von Objekten
\item
  Normalenvektoren können im (Gruppen-)Objekteditor ausgeblendet werden
\item
  PreivewHandler zeigt Readme von .l3dpack Dateien
\item
  Joystickfunktionen Zugkraftregler (+/-) und AFB-Ziel (+/-) an
  Verhalten von Fahrstufe (+/-) angeglichen
\item
  Neue Icons
\end{itemize}

\begin{itemize}
\itemsep1pt\parskip0pt\parsep0pt
\item
  Vorsichtsignal (Zs7) ermöglicht ebenfalls die Abfahrt
\item
  Anzeige Indusi-Art in Fahrtenschreiber + LokInfo Anzeige korrigiert
\item
  Installer räumt Registry- und Startmenüeinträge bzw Dateien des alten
  Installers (2.7.2) auf
\item
  Abstürze des WetterEditors behoben
\item
  Streckensound im selben Ordner wie .l3dstr-Datei nun möglich
\item
  Ende Fahrt Anzeige überdeckt nicht mehr Fahrtauswertung
\item
  Diverse Tippfehler korrigiert
\item
  Bugfix für Anzeige Schnellbremsung bei OLE bzw. TCP
\item
  keine Soundkarte Fehlermeldung wird nur 1x gezeigt
\item
  Absturz LoksimEdit bei Verwendung von Touch-Monitor behoben
\item
  Fehler bei F11 bei kurzen Haltabständen behoben
\item
  Fehler bei Planabfahrt kurz nach 00:00 behoben
\end{itemize}

\paragraph{Version 2.8}

11. März 2012

\begin{itemize}
\itemsep1pt\parskip0pt\parsep0pt
\item
  Zusätzliche und erweiterte Signalfunktionen
\item
  Geschwindigkeitsprüfabschnitte
\item
  Logische Ausdrücke zur Sichtbarkeitssteuerung bzw Soundsteuerung
\item
  Wetter- / Himmelstextursteuerung
\item
  In Fahrplänen kann ein Sound definiert werden, welcher in einer
  bestimmten Entfernung vor einem (Bedarfs)halt abgespielt wird
\item
  Standarddateidialog auch unter Windows XP verwendbar
\item
  Neuer PackageManager für die (De)installation von Loksim-Packages
\item
  BÜ-Namen können nicht nur aus einer Liste ausgewählt werden, sondern
  auch direkt per Namen eingegeben werden (sinnvoll falls BÜ in anderer
  Streckendatei definiert ist)
\item
  Vertikaler Schriftzug bei Gruppenobjekten
\item
  Neue Hauptsignaleigenschaft als Checkbox: Zwischensignal
\item
  Einbau CH-Sifa mit und ohne HS-Auslösung. Minor Bugfixes CH-Indusi und
  neu Auslösung mit oder ohne HS-Auslösung.
\item
  Fehlende Verzögerung zwischen Schalten von BueLicht und BueSchranke
  bei Bahnübergängen eingebaut
\item
  {[}Vista+{]} Dateivorschau Handler (PreviewHandler) wird bei
  Installation registriert
\item
  {[}Vista+{]} Aufnahme der Loksim-Dateien in Windows-Suchindex bei
  Installation
\item
  Programmende beim Überfahren eines Rot zeigenden Signales ist
  abschaltbar
\item
  Erweiterung des Kombihbels um die Option "nur dyn. Bremse".
\item
  Fehlerkorrektur: Keine vZielüberwachung in der PZB bei Welchselblinken
  ohne 500er und 1000er-Melder.
\item
  Fehlerkorrektur: Nullstellungszwangauflösung über Zugkraft 0 möglich.
\item
  Joysticksteuerung bei Beschl+Bremsen mit Zugkraftregler korrigiert
\item
  Endlosschleifen KBS-Editor Weichenstellung verhindert
\item
  Bei der Zifferneingabe im Editor werden jetzt auch Kommas (Beistriche)
  akzeptiert
\item
  Real-Sifataste auch auf den Achsen der Joysticksteuerung
\item
  Bugfix beim Bestimmen des relativen Pfads von Loksim-Dateien gegenüber
  anderen Loksim-Dateien
\item
  Die Ausgabe 'Streckenlimit' so gesetzt, das auch ein Limit von 99km/h
  die Kennziffer 9 ergibt.
\item
  Standard-3D Einstellungen geändert: Max-Texturgröße 1024, Alle
  Texturen Laden und im Speicher halten, Hohe Farbtiefe, Cache nicht
  verwenden
\item
  L3dEditLauncher startet bei Vorhandensein mehrerer
  Loksim-Installationen immer den Editor im richtigen Ordner
\item
  Unterstützung von CrashDumps
\item
  Einsatz des SSE2 Befehlssatz bei neueren CPUs
\end{itemize}

\paragraph{Version 2.7.2}

12. Dezember 2010

\begin{itemize}
\itemsep1pt\parskip0pt\parsep0pt
\item
  Bug Lok Editor behoben (fehlende Eingabefelder für Instrumente)
\end{itemize}

\paragraph{Version 2.7.1}

29. November 2010

\begin{itemize}
\itemsep1pt\parskip0pt\parsep0pt
\item
  Erweiterte Bahnübergangssteuerung
\item
  Es kann der Sifa/Indusi-Zwangsbremsungssound mit der Option "nur
  einmal" versehen werden.
\item
  Auswertungs Bitmap funktioniert nun auch bei GDI Darstellung
\item
  Bug bei Laufleistung km Protokollierung behoben (alte Statistiken
  werden automatisch gelöscht)
\item
  Button Texturecache löschen funktioniert wieder
\item
  Streckensounds bei mehreren Modulen funktioniert nun
\item
  Bug in der Fahrtenschreiberauswertung behoben (- Prozente)
\item
  Bug bei Störungshäufigkeit von Signalgedeckten und Signalgedeckt
  (Streckenblock) BÜs
\item
  Standard Dateiinfo und -autor kann angegeben werden
\item
  Ab Vista kann anstatt dem loksimspezifischem der Standard Windows
  Dialog verwendet werden
\item
  Anzeige von Limits bei denen "Limit nicht anzeigen" gesetzt ist,
  werden standardmäßig in der EBULA nicht mehr angezeigt. Für Anfänger
  gibt es die Möglichkeit diese realitätssteigernde Option auszuschalten
\item
  Infofeld bei Fahrplan- und Lokauswahl vergrößert
\item
  Bug Haltestellenansage nach letztem Halt behoben
\item
  Bug Zs9 Meldung kommt nur mehr bei gestörten BÜs
\item
  PZB Befehl auf Maustaste möglich
\item
  Unnötige Änderungen übernehmen Nachfrage im Editor bei Weitsichtbar
  behoben
\item
  Streckensounds im Editor standardmäßig stummgeschaltet; Zustand der
  "Stummschaltung" wird bei Ein- und Ausschalten der Vorschau übernommen
\item
  Einbau SBB-Signum
\item
  Bug Uninstall behoben
\item
  Farbe und Breite für Sekundenzeiger in Analoguhr einstellbar
\item
  Bei Package-Installation werden die Zeitstempel bei versuchtem
  Überchreiben von schreibgeschützten Dateien angezeigt
\item
  Bug bei Zs1 behoben
\end{itemize}

\paragraph{Version 2.7}

1. Juli 2010

\begin{itemize}
\itemsep1pt\parskip0pt\parsep0pt
\item
  2D Darstellung standardmäßig mit DirectX
\item
  Möglichkeit Sounds auf Strecke einzubinden
\item
  Fehlerkorrekturen bei Packageinstallation
\end{itemize}

