\section{Übersicht}

Der PackageManager wird zur (De)installation von Loksim Packages
verwendet. Loksim Packages besitzen im Normalfall die Dateiendung
.l3dpack (teilweise auch .zip).

Im Normalfall wird beim Öffnen einer .l3dpack Datei automatisch der
PackageManager gestartet. Ein Klick auf Installation starten startet die
Installation. Wird ein Package nicht mehr benötigt, kann es im Tab
Packages deinstallieren wieder von der Festplatte gelöscht werden. Dabei
werden eventuelle Abhängigkeiten von anderen Packages beachtet und
auschließlich nicht mehr benötigte Dateien gelöscht

\section{Funktionsweise}

\begin{itemize}
\itemsep1pt\parskip0pt\parsep0pt
\item
  Bei jeder Installation eines Packages protokolliert der Manager welche
  Dateien installiert wurden. So kann er bestimmen, welche Dateien von
  welchen Packages benötigt werden. Bei der Deinstallation eines
  Packages werden jene Dateien gelöscht, die ansonsten von keinem
  Package mehr benötigt werden. Dateien die bereits im
  Loksim-Verzeichnis existieren, werden niemals gelöscht (es ist im
  Nachhinein nicht bestimmbar, welche Packages welche Dateien benötigen;
  bzw welche Packages überhaupt installiert sind)
\item
  Wie bisher werden bestehende Dateien in den Backup Ordner kopiert,
  falls sie bei einer Package Installation überschrieben werden. Bei
  einer Deinstallation werden sämtliche gelöschten Dateien ebenfalls in
  den Backup Ordner verschoben
\item
  Bei der Deinstallation von Packages muss man sich im Klaren sein, dass
  nicht exakt der Zustand "vor der Installation" wiederhergestellt wird:
  Beispiel: Man installiert das Package A und dann B. Beide installieren
  die Datei x (im Package B ist die Datei x neuer). Bei der
  Deinstallation von Package B bleibt jedoch die Datei x, die bei der
  Installation von Package B kopiert wurde, zurück.
\item
  Über die Optionen ist die Funktion  \emph{Installation rückgängig machen} aktivierbar: Im Gegensatz zur Deinstallation eines Package, kopiert
  diese Funktion die gesicherten überschriebenen Dateien aus dem Backup
  Verzeichnis zurück in das Loksim-Verzeichnis. Jedoch ist diese
  Funktion immer nur für das zuletzt installierte Package anwendbar und
  nicht für früher installierte Packages.

 \emph{Wichtig:  Deinstallierte Packages können nicht wiederhergestellt werden!}
\end{itemize}

\section{Deinstallation während Installation}\abversion{2.8.2}

Wie im Abschnitt über \hyperref[sec:editor.allg.packages]{Package
erzeugen} beschrieben wird, können bei der Installation eines neuen
Package gleichzeitig ältere Packages deinstalliert werden.

Die Motivation dahinter ist, dass es oftmals neuere Versionen von
Packages gibt wo sich die Ordnerstrukturen ändern, manche Objekte nicht
mehr gebraucht werden oder Duplikate gelöscht wurden. Installiert der
Benutzer diese neue Version, bleiben jedoch die alten, nicht mehr
gebrauchten Dateien trotzdem im Loksim Verzeichnis zurück. Bei
Führerständen und Fahrplänen kann dies sogar zur Verwirrung des
Benutzers führen, bei sämtlichen anderen Dateien bleiben unschöne
''Leichen'' im Loksim Verzeichnis zurück die nicht mehr gebraucht werden

Wird jedoch beim Erzeugen des Package darauf geachtet, dass sämtliche
älteren Versionen des Package bei der Installation der neuen Version
gelöscht werden, gibt es solche Probleme nicht. Technisch betrachtet
verhält sich die Deinstallation eines Package während der Installation
eines anderen Package so, als würde man zuvor manuell die Deinstallation
der älteren Packages vornehmen. Jedoch ist der Mechanismus etwas
ausgewachsener, sodass wirklich nur jene Dateien gelöscht bzw. kopiert
werden, bei denen es tatsächlich auch nötig ist. So sind in der
Übersicht der installierten Dateien wirklich nur neuere Versionen zu
sehen

\emph{Einschränkungen:} Bei dieser Art von Deinstallation ist es nicht
möglich auszuwählen, welche Dateien exakt deinstalliert werden sollen.
Packages werden hierbei ganz oder gar nicht deinstalliert. Außerdem
bleibt das Prinzip erhalten, dass Deinstallationen nicht rückgängig
gemacht werden können! Weder über die Package deinstallieren, noch über
die Installation rückgängig machen Funktion
