\section{Darstellung}
\label{sec.sim.optionen.darstellung}
\begin{description}
\item[Faktoren Objekte ausblenden] Diese Option steuert wie schnell Objekte ausgeblendet werden um die fps zu erhöhen. Es wird empfohlen den Standardwert \emph{1} gesetzt zu lassen. Werte zwischen 0 und 1 führen dazu, dass Objekte später ausgeblendet werden, also länger sichtbar sind. Bei Werten größer als 1 werden Objekte früher ausgeblendet und die fps steigen. Die Option kann für normale Objekte und für Objekte bei welchen \emph{weit sichtbar} definiert ist getrennt gesetzt werden.\abversion{2.9}
\item[Windows 8 Vollbildmodus] Bis Version 2.8.3 war die Performance von Loksim unter Windows 8 im Vollbildmodus auf vielen Systemen im Vergleich zur Performance im Fenstermodus sehr schlecht. Mit Version 2.9 wurde deshalb die Option \emph{Windows 8 Vollbildmodus} eingeführt. Diese Option kann die Vollbildperformance unter Windows 8 erheblich verbessern, bei manchen Systemen führt sie jedoch zu Bildfehlern. Windows 8 Nutzer sollte diese Option bei Bedarf auf dem eigenen System testen. Mit dieser aktivierten Option gibt es jedoch ein anderes Problem:
Dialogboxen des Loksim werden im Vollbildmodus mit aktivierter ''Windows 8 Vollbildmodus'' Option nicht im Vordergrund angezeigt.
Dialogboxen wie ''Loksim wirklich beenden?'' oder ''Halt überfahren - Zurück zu Halt?'' müssen deshalb ''blind'' mit den Cursor-Tasten und Enter bedient  werden.\abversion{2.9}


\end{description}

\section{Simulation}
\label{sec.sim.optionen.simulation}
\begin{description}
\item[Doppelklick auf Sifa-Taste startet Wegmessung] Diese Option ist für Loks relevant, welche mit einem Zuglängenzahler / Wegmessung ausgestattet sind. Falls diese Einstellung aktiviert ist, kann die Wegmessung mit zweimaligem Drücken der Sifa-Taste oder zweimaligem Loslassen der realen Sifa-Taste (Joysticksteuerung), gestartet werden.\abversion{2.8.3}
\end{description}

\section{Konfigurationsdateien}
\subsection{Datenpfade}
\label{sec:sim-optionen-paths}
Loksim unterscheidet zwischen zwei grundsätzlichen Orten an denen wichtige Dateien abgelegt werden:
\begin{description}
\item[Programmverzeichnis] Hier werden die ausführbaren Programmdateien und einige essentielle sonstige Dateien die unbedingt zum Betrieb benötigt werden abgelegt. Standardmäßig befindet sich dieses Verzeichnis unter Programm-Verzeichnis/Loksim3D
\item[Datenverzeichnis] In diesem Verzeichnis werden Addons/Packages bzw. auch das Loksim-Standardpackage abgelegt. Standardmäßig befindet sich dieses Verzeichnis unter Öffentliche-Dokumente/Loksim3D
\end{description}
Beide Verzeichnisse können bei der erstmaligen Installation von Loksim3D eingestellt werden und mittels ''Über Loksim'' angezeigt werden. Es ist auch möglich, dass beides auf das gleiche Verzeichnis verweißt.

Im selben Verzeichnis wie Loksim3D.exe bzw. LoksimEdit.exe (Programmverzeichnis) wird für diesen Zweck eine Datei paths.ini abgelegt. In dieser wird das Datenverzeichnis angegeben. Hierbei sind sowohl relative als auch absolute Pfadangaben möglich.

Alternativ kann auch ''<Registry>'' angegeben werden: Dann wird der Pfad aus der Registry ausgelesen, welcher bei der letzten Installation von Loksim3D als Datenverzeichnis angegeben wurde.\abversion{2.9.3}