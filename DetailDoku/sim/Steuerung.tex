\section{Diverses}
\label{sec.sim.steuerung.diverses}
\subsection{Zuglängenzähler / Wegmessung}\abversion{2.8.3}
\label{sec.sim.steuerung.diverses.wegmessung}
Moderne Loks sind häufig mit einem Zuglängenzähler (auch als Wegmessung bekannt) ausgerüstet. Einsatzgebiet ist beispielsweise das Ende einer Langsamfahrstelle: Sobald das Tfz das Ende des Gefahrenbereichs erreicht, startet der Tfzf die Wegmessung. Nachdem die Lok eine Zuglänge zurückgelegt hat, ertönt ein akkustisches Signal. Dadurch weiß der Tfzf, dass es sicher ist, den Zug zu beschleunigen.

Falls ein Loksim-Führerstand mit dem Zuglängenzähler ausgestattet ist, kann dieser über die in den Optionen einstellbare Tastenkombination, gestartet werden. Sobald der Zug die Zuglänge abgefahren hat, ertönt ein Ton. Optional kann auch beim Start der Wegmessung ein akkustisches Signal erfolgen. Daneben kann die Wegmessung auch mit einem Doppelklick auf die Sifa-Taste bzw. zweimaligem Loslassen der Sifa-Taste (reale Sifa-Taste Joysticksteuerung) gestartet werden. Dieses Verhalten ist in den \hyperref[sec.sim.optionen.simulation]{Optionen} abschaltbar.